% This document is under construction

\documentclass{article}
\usepackage[utf8]{inputenc}
\usepackage{hyperref}
\usepackage{graphicx}
\usepackage{multirow}

\begin{document}

\title{Solution to the exercises of the book ``Introduction to Rocket Science and Engineering''}
\author{Arturo Gonzalez\\
	\texttt{\href{mailto:arturo.gonzalez@argonur.com}{arturo.gonzalez@argonur.com}}}
\date{\today}
\maketitle

\begin{abstract}
This document contains the solution to the exercises of each chapter of the book ``Introduction to Rocket Science and Engineering'' by Travis S. Taylor. Solving the exercises is a project proposed in the blog \url{https://www.argonur.com}.\\
To visit the website of the project: \\
\url{https://argonur.com/solucion-introduction-to-rocket-science-and-engineering/}. 
\end{abstract}

\cleardoublepage

%%%%%%%%%%%%%%%%%%%%%%%%%%%%%
%					Chapter 1											 %
%%%%%%%%%%%%%%%%%%%%%%%%%%%%%

%\section{Exercises of Chapter 1}

\begin{enumerate}
	\item {\bf Discuss the relevance of the \textit{aelopile} to rocket science and why it was considered the first demonstration of the principle of rocketry.}\\

	\item {\bf What are the main components of the gun powder?}\\
	
	\item {\bf What was \textit{Principia} and why is it relevant to rocket science?}\\
	
	\item {\bf Why were William Hale’s rockets “better” than William Congreve’s?}\\
	
	\item {\bf Compare and contrast the contributions to the development of rocketry by Konstantin Tsiolkovsky and Robert Goddard. Which one could be considered the “father of rocket science” and which one the “father of rocket engineering”?}\\
	
	\item {\bf Who was known as the Chief Designer and why?}\\
	
	\item {\bf Who was the Chief Designer’s counterpart in the American space program?}\\
	
	\item {\bf What is the oldest spacecraft still in orbit?}\\
	
	\item {\bf What is UDMH? What is it used for? What is NTO?}\\
	
	\item {\bf Draw a simple liquid fuel rocket and label all the major subcomponents.}\\
	
\end{enumerate}

\section{Exercises of Chapter 1}

\begin{enumerate}
	\item {\bf Discuss the relevance of the \textit{aelopile} to rocket science and why it was considered the first demonstration of the principle of rocketry.}\\

The aeolipile consists of a vessel, usually a "simple" solid of revolution, such as a sphere or a cylinder, arranged to rotate on its axis, having oppositely bent or curved nozzles projecting from it (tipjets). When the vessel is pressurised with steam, steam is expelled through the nozzles, which generates thrust due to the rocket principle as a consequence of the 2nd and 3rd of Newton's laws of motion. When the nozzles, pointing in different directions, produce forces along different lines of action perpendicular to the axis of the bearings, the thrusts combine to result in a rotational moment (mechanical couple), or torque, causing the vessel to spin about its axis. \cite{aeolipile}

	\item {\bf What are the main components of the gunpowder?}\\
	
Gunpowder, also known as black powder, is the earliest known chemical explosive. It is a mixture of sulfur, charcoal, and potassium nitrate (saltpeter). The sulfur and charcoal act as fuels, and the saltpeter is an oxidizer. Because of its burning properties and the amount of heat and gas volume that it generates, gunpowder has been widely used as a propellant in firearms, as a pyrotechnic composition in fireworks and as a blasting powder in quarrying, mining, and road building. \cite{gunpowder}
	
	\item {\bf What was \textit{Principia} and why is it relevant to rocket science?}\\
	
	\item {\bf Why were William Hale’s rockets “better” than William Congreve’s?}\\
	
	\item {\bf Compare and contrast the contributions to the development of rocketry by Konstantin Tsiolkovsky and Robert Goddard. Which one could be considered the “father of rocket science” and which one the “father of rocket engineering”?}\\
	
	\item {\bf Who was known as the Chief Designer and why?}\\
	
	\item {\bf Who was the Chief Designer’s counterpart in the American space program?}\\
	
	\item {\bf What is the oldest spacecraft still in orbit?}\\
	
	\item {\bf What is UDMH? What is it used for? What is NTO?}\\
	
	\item {\bf Draw a simple liquid fuel rocket and label all the major subcomponents.}\\
	
\end{enumerate}

%%%%%%%%%%%%%%%%%%%%%%%%%%%%%
%					End Chapter 1											 %
%%%%%%%%%%%%%%%%%%%%%%%%%%%%%

\cleardoublepage

\begin{thebibliography}{99}

\bibitem{book}
	Travis S. Taylor,
	\emph{Introduction to Rocket Science and Engineering}.
	CRC Press,
	2009.
	
\bibitem{aeolipile}
	Aeolipile\\
	Wikipedia, the free encyclopedia\\
	April 2017\\
	\url{https://en.wikipedia.org/wiki/Aeolipile}

\bibitem{gunpowder}
	Gunpowder\\
	Wikipedia, the free encyclopedia\\
	April 2017\\
	\url{https://en.wikipedia.org/wiki/Gunpowder}

\end{thebibliography}

\end{document}
